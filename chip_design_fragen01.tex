% ----------------------- TODO ---------------------------
%Template 
\documentclass[a4paper]{scrartcl}
\usepackage[utf8]{inputenc}
%\usepackage[ngerman]{babel}
\usepackage{geometry,forloop,fancyhdr,fancybox,lastpage}
\usepackage{listings}
\lstset{frame=tb,
	language=Java,
	aboveskip=3mm,
	belowskip=3mm,
	showstringspaces=false,
	columns=flexible,
	basicstyle={\small\ttfamily},
	numbers=left,
	numberstyle=\tiny\color{gray},
	keywordstyle=\color{blue},
	commentstyle=\color{dkgreen},
	stringstyle=\color{mauve},
	breaklines=true,
	breakatwhitespace=true,
	tabsize=3
}
\geometry{a4paper,left=3cm, right=3cm, top=3cm, bottom=3cm}
% Diese Daten müssen pro Blatt angepasst werden:
\newcommand{\NUMBER}{1}
\newcommand{\EXERCISES}{11}
% Diese Daten müssen einmalig pro Vorlesung angepasst werden:
\newcommand{\COURSE}{Chip Design}
%\newcommand{\TUTOR}{Benjamin Coban}
\newcommand{\STUDENTA}{Stefan Wezel}
\newcommand{\STUDENTB}{Lukas Günthner}
%\newcommand{\STUDENTC}{Gwent Krause}
\newcommand{\DEADLINE}{\date}
% ----------------------- TODO ---------------------------



%Math
\usepackage{amsmath,amssymb,tabularx}

%Figures
\usepackage{graphicx,tikz,color,float}
\graphicspath{ {home/stefan/picures/} }
\usetikzlibrary{shapes,trees}

%Algorithms
\usepackage[ruled,linesnumbered]{algorithm2e}

%Kopf- und Fußzeile
\pagestyle {fancy}
%\fancyhead[L]{Tutor: \TUTOR}
\fancyhead[C]{\COURSE}
\fancyhead[R]{\today}

\fancyfoot[L]{}
\fancyfoot[C]{}
\fancyfoot[R]{Seite \thepage}

%Formatierung der Überschrift, hier nichts ändern
\def\header#1#2{
	\begin{center}
		{\Large\bf Übungsblatt #1}\\
		{(Abgabetermin #2)}
	\end{center}
}

%Definition der Punktetabelle, hier nichts ändern
\newcounter{punktelistectr}
\newcounter{punkte}
\newcommand{\punkteliste}[2]{%
	\setcounter{punkte}{#2}%
	\addtocounter{punkte}{-#1}%
	\stepcounter{punkte}%<-- also punkte = m-n+1 = Anzahl Spalten[1]
	\begin{center}%
		\begin{tabularx}{\linewidth}[]{@{}*{\thepunkte}{>{\centering\arraybackslash} X|}@{}>{\centering\arraybackslash}X}
			\forloop{punktelistectr}{#1}{\value{punktelistectr} < #2 } %
			{%
				\thepunktelistectr &
			}
			#2 &  $\Sigma$ \\
			\hline
			\forloop{punktelistectr}{#1}{\value{punktelistectr} < #2 } %
			{%
				&
			} &\\
			\forloop{punktelistectr}{#1}{\value{punktelistectr} < #2 } %
			{%
				&
			} &\\
		\end{tabularx}
	\end{center}
}

\begin{document}
	
	\begin{tabularx}{\linewidth}{m{0.2 \linewidth}X}
		\begin{minipage}{\linewidth}
			\STUDENTA\\
			\STUDENTB\\
			%\STUDENTC
		\end{minipage} & \begin{minipage}{\linewidth}
			\punkteliste{1}{\EXERCISES}
		\end{minipage}\\
	\end{tabularx}
	
	%\header{Nr. \NUMBER}{\DEADLINE}
	
	% ----------------------- TODO ---------------------------
	% Hier werden die Aufgaben/Lösungen eingetragen:
	
	

\section*{Augabe 1}
Moore's Law besagt, dass sich ca. alle zwei Jahre die Anzahl an Transistoren verdoppelt. Sie entstand durch Beobachtung der Vergangenheit und Projektion dieser Beobachtungen auf die Zukunft. Demnach ist Moore's Law nicht physikalisch begründert.



\section*{Aufgabe 2}
Scaling ist ein Begriff für die Weiterentwicklung elektronischer Bauelemente im Sinn von:
\begin{itemize}
	\item Strukturverkleinerung
	\item Erhöhung der Komplexität
	\item Verringerung der Kosten pro implementierte Funktion
	\item Verringerung der Leisutngsaufnahme pro Operation
\end{itemize}
Daraus folgt jedoch:
\begin{itemize}
	\item Steigender Fertigungsaufwans für Wafer
	\item Höhere Anforderung an Fotolithographie
	\item Erhöhung des Entwurfsaufwands für ein Produkt
\end{itemize}

\section*{Aufgabe 3}
Pitch ist die Summe von minimaler Breite und minimalem Abstand der Masken-Ebene.


\section*{Aufgabe 4}
Kleiner IC's führt zu größerer Ausbeute. Desweitern ist man durch kleinere Tranistoren in der Lage im selben Platzverbrauch mehr Rechenleistung zu erreichen.


\section*{Augabe 5}
Hybridschaltungen sind Kombinationen aus (meist) mehreren Halbleiterbauelementen mit keramischen Trägerplatte. Die Komponenten werden teils gemeinsam hergestellt und teils auf das Keramik substrat montiert. Das heißt es werden beispielsweise auf einer Kreramikträgerplatte IC's, Dioden und Einzeltransistoren aufgebracht, die die gewünschte Funktionsweise der Schaltung realisieren.


\section*{Aufgabe 6}
Monolithische Integration bedeutet, dass alle Komponenten eines IC's gemeinsam hergestellt werden. Es wird ein einzelner Halbleiter-Einkristall als Substrat verwendet auf dem der ganze IC realisiert ist.


\section*{Aufgabe 7}
Ein Fin-Fet ist eine dreidimensionale Bauweise für ein MOS-Transistor. Bei Fin-Fets wird das Gate um die Drain und Source umschlossen, das heißt es nun, im Vergleich zur planaren Bauweise, nun drei Seiten durch das Gate abgedeckt. Dadurch ist man in der Lage den leitenden Kanal zu vergrößern.\\
Ein nanowire-Transistor ist eine weitere dreidimesionale Bauweise für MOS-Transistoren. Man verwendet hier die Gate-All-Around Bauweise. Dadurch ist man in der Lage das gate komplett um Source und Drain zu umschliessen. Demenetsprechend wird der geöffnete leitende Kanal noch größer als bei Fin-Fets. Durch vergrößern des leitenden Kanals ist man in der Lage die Verlustleistung signifikant zu reduzieren.


\section*{Aufgabe 8}



\section*{Augabe 9}
Seit einigen Jahren ist nicht sicher wie weit die Skalierung noch gehen kann. Deshalb werden verschiedene Wege gegangen. Es wird deshalb versucht nicht mehr nur die geometrische Größe von Transistoren zu reduzieren, sondern wird auch versucht die Funktionalität und Effizienz von Chips (wie Sensoren und nichtflüchtigem Speicher)zu steigern.


\section*{Aufgabe 10}
Bei CMOS wird komplementär ein n-Kanal und p-Kanal (mit Pullup/Pulldown-Netzwerken) MOS-Fet aufgebaut.


\section*{Aufgabe 11}
\begin{itemize}
	\item Geringere Leistungsaufnahme, da geringer Querstrom (nur Leckstrom)
	\item Großer Störabstand und dementsprechende günstige Übertragungskennlinie
	\item Hohe Unempfindlichkeit gegen Herstellungs-Parameterstreuungen. Pegel nich von der ratio der Leitfähigkeit von Treibtransistor zu Lasttransistor abhängig.
\end{itemize}





\end{document}
