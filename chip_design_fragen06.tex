 % ----------------------- TODO ---------------------------
%Template 
\documentclass[a4paper]{scrartcl}
\usepackage[utf8]{inputenc}
%\usepackage[ngerman]{babel}
\usepackage{geometry,forloop,fancyhdr,fancybox,lastpage}
\usepackage{listings}
\lstset{frame=tb,
	language=Java,
	aboveskip=3mm,
	belowskip=3mm,
	showstringspaces=false,
	columns=flexible,
	basicstyle={\small\ttfamily},
	numbers=left,
	numberstyle=\tiny\color{gray},
	keywordstyle=\color{blue},
	commentstyle=\color{dkgreen},
	stringstyle=\color{mauve},
	breaklines=true,
	breakatwhitespace=true,
	tabsize=3
}
\geometry{a4paper,left=3cm, right=3cm, top=3cm, bottom=3cm}
% Diese Daten müssen pro Blatt angepasst werden:
\newcommand{\NUMBER}{1}
\newcommand{\EXERCISES}{10}
% Diese Daten müssen einmalig pro Vorlesung angepasst werden:
\newcommand{\COURSE}{Chip Design}
%\newcommand{\TUTOR}{Benjamin Coban}
\newcommand{\STUDENTA}{Stefan Wezel}
\newcommand{\STUDENTB}{Lukas Günthner}
%\newcommand{\STUDENTC}{Gwent Krause}
\newcommand{\DEADLINE}{\date}
% ----------------------- TODO ---------------------------



%Math
\usepackage{amsmath,amssymb,tabularx}

%Figures
\usepackage{graphicx,tikz,color,float}
\graphicspath{ {home/stefan/picures/} }
\usetikzlibrary{shapes,trees}

%Algorithms
\usepackage[ruled,linesnumbered]{algorithm2e}

%Kopf- und Fußzeile
\pagestyle {fancy}
%\fancyhead[L]{Tutor: \TUTOR}
\fancyhead[C]{\COURSE}
\fancyhead[R]{\today}

\fancyfoot[L]{}
\fancyfoot[C]{}
\fancyfoot[R]{Seite \thepage}

%Formatierung der Überschrift, hier nichts ändern
\def\header#1#2{
	\begin{center}
		{\Large\bf Übungsblatt #1}\\
		{(Abgabetermin #2)}
	\end{center}
}

%Definition der Punktetabelle, hier nichts ändern
\newcounter{punktelistectr}
\newcounter{punkte}
\newcommand{\punkteliste}[2]{%
	\setcounter{punkte}{#2}%
	\addtocounter{punkte}{-#1}%
	\stepcounter{punkte}%<-- also punkte = m-n+1 = Anzahl Spalten[1]
	\begin{center}%
		\begin{tabularx}{\linewidth}[]{@{}*{\thepunkte}{>{\centering\arraybackslash} X|}@{}>{\centering\arraybackslash}X}
			\forloop{punktelistectr}{#1}{\value{punktelistectr} < #2 } %
			{%
				\thepunktelistectr &
			}
			#2 &  $\Sigma$ \\
			\hline
			\forloop{punktelistectr}{#1}{\value{punktelistectr} < #2 } %
			{%
				&
			} &\\
			\forloop{punktelistectr}{#1}{\value{punktelistectr} < #2 } %
			{%
				&
			} &\\
		\end{tabularx}
	\end{center}
}

\begin{document}
	
	\begin{tabularx}{\linewidth}{m{0.2 \linewidth}X}
		\begin{minipage}{\linewidth}
			\STUDENTA\\
			\STUDENTB\\
			%\STUDENTC
		\end{minipage} & \begin{minipage}{\linewidth}
			\punkteliste{1}{\EXERCISES}
		\end{minipage}\\
	\end{tabularx}
	
	%\header{Nr. \NUMBER}{\DEADLINE}
	
	% ----------------------- TODO ---------------------------
	% Hier werden die Aufgaben/Lösungen eingetragen:
	
	

\section*{Frage 1}
Im allgemeinen besteht ein komplementärsymmetrisches Gatter aus einem p-Kanal pull-up Schaltnetz und einem der zu realisierenden logischen Funktion entsprechenden n-Kanal pull-down Schaltnetz. Komplementärsymmetrie bedeutet, dass beide (n- und p-Kanal) Schaltnetze die selbe logische Funktion realisieren.
~\\
~\\
\section*{Frage 2}
Es wird CMOS bevorzugt, da CMOS einen geringeren Energieverbrauch hat und es kaum ein statischen Querstrom gibt. Zum anderen hat CMOS eine größere Stör-Unempfindlichkeit, da CMOS Gatter eine günstige Gleichspannungs-Übertragungskennlinie haben. Der wichtigste elektrische Vorteil ist der geringere Energieverbrauch, da so zum einen die Leistungsaufnahme für Chips kleiner ist und zum anderen weniger Abwärme produziert wird. So können die Chips auch mit einer höheren Taktfrequenz betrieben werden. Der geringe Energieverbrauch resultiert aus dem kaum vorhandenen Querstrom, es fließt nur beim umladen der Kapazitäten, also beim schalten, ein starker Stromstoß der die Lastkapazität umläd.
~\\
~\\
\section*{Frage 3}
$P = \frac{1}{2} \cdot C \cdot V_{dd}^2 \cdot f_{eff} + P_{Leck}$\\
~\\
mit $P = \text{ Verlustleistung}$, $C = \text{ Kapazität}$, $V_{dd} = \text{ Versorgungsspannung}$, $f_{eff} = \text{ Bit-Wechsel-Rate}$ und $P_{Leck} = \text{ Leistung des Leckstroms}$.
~\\
~\\
\section*{Frage 4}
Die High- oder Low-Ausgangsspannung ist nicht vom Verhältnis der Treiberstärken der Pull-Up oder Pull-Down Transistoren abhängig. Deshalb ist die Ausgansspannung auch nicht von der Streuung der Tranistorparameter abhängig solange der jeweilige Transistor hinreichend gesperrt ist und bleibt. Es ist jedoch zu beachten, dass es keine zu hohe Leckströme gibt und keine exzessive Verschiebung der Schwellspannung.
~\\
~\\
\section*{Frage 5}
Es werden grundsätzlich n- und p-Kanaltranistoren paarweise verwendet. Entweder das n-Netz oder das p-Netz ist eingeschalten, niemals beide gleichzeitig (abgesehen im Schaltvorgang). Desweiteren fließt kein Dauerstrom abgesehen von Leckströmen. Zuletzt werden keine dynamischen Schaltungen verwendet, das heißt die minimale Taktfrequenz ist $f_{Tmin} = 0 Hz$.
~\\
~\\
\section*{Frage 6}
Nein, da das Pull-Up Netz gegen die Versorgungsspannung geschalten ist und N-MOS Transistoren gegen GND geschalten werden müssen.
~\\
~\\
\section*{Frage 7}

~\\
~\\
\section*{Frage 8}
Wenn die Komplexität erhöht wird, bedeutet dies für die Fertigung, dass sie mehr Gatter auf einem Chip unterbringen muss. Dafür gibt es unterschiedliche Möglichkeiten. Prinzipiell spielt Platz dabei eine große Rolle. Es müssen möglichst viele Gatter auf möglichst wenig Platz untergebracht werden.\\
Für die Fertigung kann es dabei hilfreich sein, sich nicht komplett auf automatische Verfahren zu verlassen sondern zumindest Teilweise die Schaltpläne von Hand anfertigen zu lassen. Auf diese Weise lässt sich Platz sparen, und es kann die Komplexität erhöht werden, ohne mehr Platz zu verbrauchen.
~\\
~\\
\section*{Frage 9}
\begin{itemize}
	\item $p_1$: An diesem Punkt ist das Einganssignal $U_{in} = 0V$ und das Ausganssignal $U_{out}$ logisch high.
	\item $p_2$: Hier endet die Noise Margin Zone für den logischen low output.
	\item $p_3$: Hier ist die Schwellspannung des Inverters, $U_{in} = U_{out}$.
	\item $p_4$: Hier beginnt die Noise Margin Zone für den logischen high outpu.
	\item $p_5$: An diesem Punkt ist das Eingangssignal logisch high und das Ausgangssignal $U_{out} = 0V$
\end{itemize}

Die fünf charakteristischen Abschnitte sind: 
\begin{itemize}
	\item 1: Die Spannung, die am Inverter anliegt ist $0V$. n-mos befindet sich im Cut-Off- und p-mos im linearen Bereich. Der Output ist $V_{DD}$.
	\item 2: p-mos befindet sich weiterhin im linearen Bereich und n-mos nun im Sättigungsbereich.
	\item 3: Wir sind nun in der Mitte der Übertragungskurve. Sowohl p- als auch n-mos befinden sich nun im Sättigungsbereich.
	\item 4: p-mos befindet sich noch im Sättigungsbereich und n-mos im linearen.
	\item 5: Der p-mos befindet sich im Cut-Off Bereich und n-mos im linearen Bereich. Der Output $0V$.
\end{itemize}
~\\
~\\
\section*{Frage 10}
Die Noise Marin ist die maximale Summe aller elektrischen Störungen, die dem Signal überlagert sein dürfen, ohne die Funktion zu beeinträchtigen (Quelle: Skript 6-16).
~\\
Es wird meistens die Punkte an denen die Steigung der VTC $= -1$ ist.
~\\
Von der Kanallänge, da diese extrem klein ist und demnach auch kleine Veränderungen starke Auswirkungen auf die Übertragungskennlinie des Transistors haben und die Form der Übertragungskennline zu den Noise Margins führt.
~\\
~\\
\end{document}
