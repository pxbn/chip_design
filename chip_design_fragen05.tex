 % ----------------------- TODO ---------------------------
%Template 
\documentclass[a4paper]{scrartcl}
\usepackage[utf8]{inputenc}
%\usepackage[ngerman]{babel}
\usepackage{geometry,forloop,fancyhdr,fancybox,lastpage}
\usepackage{listings}
\lstset{frame=tb,
	language=Java,
	aboveskip=3mm,
	belowskip=3mm,
	showstringspaces=false,
	columns=flexible,
	basicstyle={\small\ttfamily},
	numbers=left,
	numberstyle=\tiny\color{gray},
	keywordstyle=\color{blue},
	commentstyle=\color{dkgreen},
	stringstyle=\color{mauve},
	breaklines=true,
	breakatwhitespace=true,
	tabsize=3
}
\geometry{a4paper,left=3cm, right=3cm, top=3cm, bottom=3cm}
% Diese Daten müssen pro Blatt angepasst werden:
\newcommand{\NUMBER}{1}
\newcommand{\EXERCISES}{18}
% Diese Daten müssen einmalig pro Vorlesung angepasst werden:
\newcommand{\COURSE}{Chip Design}
%\newcommand{\TUTOR}{Benjamin Coban}
\newcommand{\STUDENTA}{Stefan Wezel}
\newcommand{\STUDENTB}{Lukas Günthner}
% \newcommand{\STUDENTC}{Gwent Krause}
\newcommand{\DEADLINE}{\date}
% ----------------------- TODO ---------------------------



%Math
\usepackage{amsmath,amssymb,tabularx}

%Figures
\usepackage{graphicx,tikz,color,float}
\graphicspath{ {home/stefan/picures/} }
\usetikzlibrary{shapes,trees}

%Algorithms
\usepackage[ruled,linesnumbered]{algorithm2e}

%Kopf- und Fußzeile
\pagestyle {fancy}
%\fancyhead[L]{Tutor: \TUTOR}
\fancyhead[C]{\COURSE}
\fancyhead[R]{\today}

\fancyfoot[L]{}
\fancyfoot[C]{}
\fancyfoot[R]{Seite \thepage}

%Formatierung der Überschrift, hier nichts ändern
\def\header#1#2{
	\begin{center}
		{\Large\bf Übungsblatt #1}\\
		{(Abgabetermin #2)}
	\end{center}
}

%Definition der Punktetabelle, hier nichts ändern
\newcounter{punktelistectr}
\newcounter{punkte}
\newcommand{\punkteliste}[2]{%
	\setcounter{punkte}{#2}%
	\addtocounter{punkte}{-#1}%
	\stepcounter{punkte}%<-- also punkte = m-n+1 = Anzahl Spalten[1]
	\begin{center}%
		\begin{tabularx}{\linewidth}[]{@{}*{\thepunkte}{>{\centering\arraybackslash} X|}@{}>{\centering\arraybackslash}X}
			\forloop{punktelistectr}{#1}{\value{punktelistectr} < #2 } %
			{%
				\thepunktelistectr &
			}
			#2 &  $\Sigma$ \\
			\hline
			\forloop{punktelistectr}{#1}{\value{punktelistectr} < #2 } %
			{%
				&
			} &\\
			\forloop{punktelistectr}{#1}{\value{punktelistectr} < #2 } %
			{%
				&
			} &\\
		\end{tabularx}
	\end{center}
}

\begin{document}
	
	\begin{tabularx}{\linewidth}{m{0.2 \linewidth}X}
		\begin{minipage}{\linewidth}
			\STUDENTA\\
			\STUDENTB\\
			%\STUDENTC
		\end{minipage} & \begin{minipage}{\linewidth}
			\punkteliste{1}{\EXERCISES}
		\end{minipage}\\
	\end{tabularx}
	
	%\header{Nr. \NUMBER}{\DEADLINE}
	
	% ----------------------- TODO ---------------------------
	% Hier werden die Aufgaben/Lösungen eingetragen:
	
	

\section*{Frage 1}
Siehe Anhang.
~\\
~\\
\section*{Frage 2}
Inversion ist der Zustand eines Halbleiters, bei dem an der Oberfläche die Dichte der Minoritätsladungsträger die Dichte der Majoritätsladungsträger übersteigt. Die Inversion kommt durch elektrostatische Anziehung (Kondensator Effekt zwischen Substrat, Gate und Halbleiter-Isolator) zustande.
~\\
~\\
\section*{Frage 3}
Die Beweglichkeit von Ladungsträgern wird zum einen von der Temperatur des Halbleiters beeinflusst, zum anderen durch die Dotierung selbst (Stöße mit Fremdatomen reduzieren Beweglichkeit).
~\\
~\\
\section*{Frage 4}
Die Schwellspannung hängt von der Ladung im Gate-Isolator, der Oberflächenladungen zwischen Substrat und Isolator, den Raumladungen im Substrat sowie der Kontaktpotentiale der Materialen ab. Durch all diese Faktoren kann man in der Fertigung die Schwellspannung beeinflussen. Beispielsweiße durch die reduzierung der Oberflächenladungen im Isolator.
~\\
~\\
\section*{Frage 5}
Ja, aber nur unter sehr viel Aufwand, da ein neues Design sowie andere Materialien zum Einsatz kommen müssen.
~\\
~\\
\section*{Frage 6}
Das Shicman-Hodges Modell ist zur Simulation ausreichend, es werden jeodch Parameter wie Temperatur- und Funkelrauschen, sub-threshold Spannung sowie Effekte im hochfrequenz Bereich nicht berückstichtigt.
~\\
~\\
\section*{Frage 7}
Je geringer die Schwellspannung, desto größer der Weak Inverision Effekt (?)\\

~\\
~\\
\section*{Frage 8}
Source-Bulk, Drain-Bluk, Gate-Source, Gate-Source-Overlap, Drain-Source, Drain-Source-Overlap und die Gate-Bulk Kapazität.
~\\
~\\
\section*{Frage 9}
Die Kapazität ist abhängig von der Sperrschichtweite. Die Sperrschichtweite (Diodenmodell) nimmt mit zuhnehmender Spannung auch zu und reduziert somit die Kapazität.
~\\
~\\
\section*{Frage 10}
Mit zunehmender Spannung in Sperrichtung nimmt die Kapazität ab. ($V_R$ größer $\Rightarrow X_d$ größer $\Rightarrow C_j$ kleiner, da $C_j = \frac{\epsilon_{si}}{X_d}$ und $\epsilon_{si}$ konstant.)
~\\
~\\
\section*{Frage 11}
Von der Konzentration der schwach dotierten Seite abhängig, da diese zu höherer Raumladungszone führt.
~\\
~\\
\section*{Frage 12}
Wird die angelegte Spannung erhöht, nimmt die Kapazität zunächst ab, bis die threshold Spannung erreicht ist. Dann nimmt die Kapazität wieder zu, bis sie ein konstantes Niveau im Sättigungsbereich des Transistors erreicht hat.\\
Die $C_{gges}$ verhält sich zwar zunächst sehr ähnlich (auch sie nimmt im Verarmungsbereich ab und steigt dann bis zum Sättigungsbereich), erreicht aber entgegen zur Mos Kapazität zunächst nicht ihr Ausgangsniveau. Dies erreicht sie erst, nachdem sie nach dem Sättigungsbereich nochmals ansteigt. 
~\\
~\\
\section*{Frage 13}
Da auf jeden Fall sichergestellt werden muss, dass das Gate die Grenze zwischen Source und Drain zu Substrat auch beeinflusst, also der durch die Gate-Spannung erzeugte Kanal auch bis zu den Source und Drain Elektroden reicht. Die Überlappung sollte aber minimal gehalten werden, parasitäre Kapazitäten zu vermeiden.
~\\
~\\
\section*{Frage 14}
Der Substrateffekt beschreibt die Abhängigkeit der Treshold Spannung zwischen Substrat und Source.\\
Der Substrateffekt kann das Kurzkanal- und DIBL(Drain induced barrier lowering)-Verhalten verschlechtern.
~\\
~\\
\section*{Frage 15}
Das Substrat dient als "Quelle" für Ladungen. Bekommt das Gate eine groß genuge Spannung, werden Elektronen aus dem Bulk des Substrats vom Channel zwischen den Wannen unter dem Gate angezogen. Dies führt zu einer negativen Ladung in dieser Region, was letztendlich einen Stromfluss ermöglicht. \\
Das Substrat ist also ein integraler Bestandteil für die Funktion eines Mosfet.
~\\
~\\
\section*{Frage 16}
Sie verdreifacht sich.
~\\
~\\
\section*{Frage 17}
Bei Finfets können kleinere Gatelengen verwendet werden, als bei planaren Transistoren. Dies führt dazu, dass auf gleichem Platz mehr Transistoren untergebracht werden können.\\
Beim normalen Planar Design stößt man hingegen bereits an die physikalischen Grenzen, was Gate-Länge betrifft.\\
Auch bei Double Gate Transistoren ist der geringere Platzverbrauch die Hauptmotivation. Dies ist möglich durch, durch das Design reduzierte Short Channel Effekte . Es wird ein gutes Verhältnis der Stromstärke zwischen On/Off erzielt.\\
Ein weiter Vorteil ist auch, dass zum normalen Produktionsprozess lediglich Schritte hinzugefügt. Dies erleichtert die Umstellung auf die neue Produktserie einer Halbleiterfabrik.
~\\
~\\
\section*{Frage 18}
Da Aluminiumleitungen eine beschränkte Leitfähigkeit hat und mit immer kleineren Strukturen die Taktverzögerung immer größer (RC-Modell) wurde und Kupfer eine besser Leitfähigkeit hat.
~\\
~\\
\section*{Frage 19}
Da ein Vakuum mit einer Dielektrizitätskonstante von 1 der beste Isolator ist.
~\\
~\\

\end{document}
