% ----------------------- TODO ---------------------------
%Template 
\documentclass[a4paper]{scrartcl}
\usepackage[utf8]{inputenc}
%\usepackage[ngerman]{babel}
\usepackage{geometry,forloop,fancyhdr,fancybox,lastpage, hyperref}
\usepackage{listings}
\lstset{frame=tb,
	language=Java,
	aboveskip=3mm,
	belowskip=3mm,
	showstringspaces=false,
	columns=flexible,
	basicstyle={\small\ttfamily},
	numbers=left,
	numberstyle=\tiny\color{gray},
	keywordstyle=\color{blue},
	commentstyle=\color{dkgreen},
	stringstyle=\color{mauve},
	breaklines=true,
	breakatwhitespace=true,
	tabsize=3
}
\geometry{a4paper,left=3cm, right=3cm, top=3cm, bottom=3cm}
% Diese Daten müssen pro Blatt angepasst werden:
\newcommand{\NUMBER}{4}
\newcommand{\EXERCISES}{3}
% Diese Daten müssen einmalig pro Vorlesung angepasst werden:
\newcommand{\COURSE}{Chip Design}
\newcommand{\TUTOR}{Julia Grosse}
\newcommand{\STUDENTA}{Stefan Wezel}
\newcommand{\STUDENTB}{Lukas Günthner}
%\newcommand{\STUDENTC}{Gwent Krause}
\newcommand{\DEADLINE}{\date}
% ----------------------- TODO ---------------------------



%Math
\usepackage{amsmath,amssymb,tabularx}

%Figures
\usepackage{graphicx,tikz,color,float}
\graphicspath{ {home/stefan/picures/} }
\usetikzlibrary{shapes,trees}

%Algorithms
\usepackage[ruled,linesnumbered]{algorithm2e}

%Kopf- und Fußzeile
\pagestyle {fancy}
\fancyhead[L]{Tutor: \TUTOR}
\fancyhead[C]{\COURSE}
\fancyhead[R]{\today}

\fancyfoot[L]{}
\fancyfoot[C]{}
\fancyfoot[R]{Seite \thepage}

%Formatierung der Überschrift, hier nichts ändern
\def\header#1#2{
	\begin{center}
		{\Large\bf Übungsblatt #1}\\
		{(Abgabetermin #2)}
	\end{center}
}

%Definition der Punktetabelle, hier nichts ändern
\newcounter{punktelistectr}
\newcounter{punkte}
\newcommand{\punkteliste}[2]{%
	\setcounter{punkte}{#2}%
	\addtocounter{punkte}{-#1}%
	\stepcounter{punkte}%<-- also punkte = m-n+1 = Anzahl Spalten[1]
	\begin{center}%
		\begin{tabularx}{\linewidth}[]{@{}*{\thepunkte}{>{\centering\arraybackslash} X|}@{}>{\centering\arraybackslash}X}
			\forloop{punktelistectr}{#1}{\value{punktelistectr} < #2 } %
			{%
				\thepunktelistectr &
			}
			#2 &  $\Sigma$ \\
			\hline
			\forloop{punktelistectr}{#1}{\value{punktelistectr} < #2 } %
			{%
				&
			} &\\
			\forloop{punktelistectr}{#1}{\value{punktelistectr} < #2 } %
			{%
				&
			} &\\
		\end{tabularx}
	\end{center}
}

\begin{document}
	
	\begin{tabularx}{\linewidth}{m{0.2 \linewidth}X}
		\begin{minipage}{\linewidth}
			\STUDENTA\\
			\STUDENTB\\
			%\STUDENTC
		\end{minipage} & \begin{minipage}{\linewidth}
			\punkteliste{1}{\EXERCISES}
		\end{minipage}\\
	\end{tabularx}
	
	%\header{Nr. \NUMBER}{\DEADLINE}
	
	% ----------------------- TODO ---------------------------
	% Hier werden die Aufgaben/Lösungen eingetragen:
	
	

\section*{Augabe 1:}
\subsection*{$I)$}
Die per Wafer $= D \cdot \pi ( \frac{D}{4S} - \frac{1}{\sqrt{2S}})$, mit $D$ wafer diameter und $S$ die-size. Formel entonmmen aus \href{https://anysilicon.com/die-per-wafer-formula-free-calculators/}{https://anysilicon.com/die-per-wafer-formula-free-calculators/}.  \\
$\Rightarrow$ Die per Wafer $ = 150 \cdot \pi (\frac{150}{400} - \frac{1}{\sqrt{200}}) \approx 143$

\subsection*{$II)$}
$Y = 0,45$\\
$Y_w = e^{-\sqrt{AD}} \Leftrightarrow -\ln (Y_w) = \sqrt{AD} \Leftrightarrow D = \frac{(-\ln(Y_w))^2}{A} \Rightarrow D = \frac{(-\ln(0,35))^2}{\pi \cdot 1^2} \approx 35,08 \%$

\subsection*{$III)$}
$Y_w = e^{- \sqrt{1,5 \cdot 0,3508}} \approx 0,4841 \Rightarrow 48,41\%$\\
"gute IC's" $= 0,4841 \cdot 90 = 43$\\
Dies per Wafer $= 150 ( \frac{150}{4 \cdot 150} - \frac{1}{\sqrt{2 \cdot 150}}) \pi \approx 90$\\

\subsection*{$IV)$}
\begin{itemize}
	\item Typ A: $K_{IC} = \frac{K_{Scheibe}}{N_{gut}} = \frac{1000}{143 \cdot 0,3508} \approx 19,93$ Euro.
	\item Typ B: $K_{IC} = \frac{K_{Scheibe}}{N_{gut}} = \frac{1000}{43} \approx 23,26$ Euro.
\end{itemize}

\section*{Aufgabe 2}
\subsection*{$I)$}
Ja, durch kleiner IC's wird beispielsweise die Anzahl defekter IC's durch Punktdefekte auf dem Wafer verringert.

\subsection*{$II)$}
$\frac{K_{new}}{K_{old}} = \frac{A_{new}}{A_{old}} \cdot e^{\sqrt{A_{new} \cdot D} - \sqrt{A_{old} \cdot D}}$\\
$\Rightarrow = \frac{1}{2} \cdot e^{\sqrt{1} - \sqrt{2}} \approx 0,33 = \frac{1}{3}$\\
$\Rightarrow$ Durch halbierung der IC Fläche $A$, lassen sich die Kosten auf ein Drittel der ursprünglichen Kosten reduzieren.


\section*{Aufgabe 3}
\subsection*{$I)$}
$Y_{ges} = 0,9 \cdot 0,82 \cdot 0,95 = 0,7011$\\
$\text{Prozesskosten}_{ges} = 1200$\\
$\text{Waferkosten}_{ges} = 1200 + 200 = 1400$\\
$\Rightarrow K_{IC} = \frac{1400}{500 \cdot 0,7011} \approx 4$ Euro.



\end{document}
