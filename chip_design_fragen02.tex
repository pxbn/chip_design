 % ----------------------- TODO ---------------------------
%Template 
\documentclass[a4paper]{scrartcl}
\usepackage[utf8]{inputenc}
%\usepackage[ngerman]{babel}
\usepackage{geometry,forloop,fancyhdr,fancybox,lastpage}
\usepackage{listings}
\lstset{frame=tb,
	language=Java,
	aboveskip=3mm,
	belowskip=3mm,
	showstringspaces=false,
	columns=flexible,
	basicstyle={\small\ttfamily},
	numbers=left,
	numberstyle=\tiny\color{gray},
	keywordstyle=\color{blue},
	commentstyle=\color{dkgreen},
	stringstyle=\color{mauve},
	breaklines=true,
	breakatwhitespace=true,
	tabsize=3
}
\geometry{a4paper,left=3cm, right=3cm, top=3cm, bottom=3cm}
% Diese Daten müssen pro Blatt angepasst werden:
\newcommand{\NUMBER}{1}
\newcommand{\EXERCISES}{11}
% Diese Daten müssen einmalig pro Vorlesung angepasst werden:
\newcommand{\COURSE}{Chip Design}
%\newcommand{\TUTOR}{Benjamin Coban}
\newcommand{\STUDENTA}{Stefan Wezel}
\newcommand{\STUDENTB}{Lukas Günthner}
%\newcommand{\STUDENTC}{Gwent Krause}
\newcommand{\DEADLINE}{\date}
% ----------------------- TODO ---------------------------



%Math
\usepackage{amsmath,amssymb,tabularx}

%Figures
\usepackage{graphicx,tikz,color,float}
\graphicspath{ {home/stefan/picures/} }
\usetikzlibrary{shapes,trees}

%Algorithms
\usepackage[ruled,linesnumbered]{algorithm2e}

%Kopf- und Fußzeile
\pagestyle {fancy}
%\fancyhead[L]{Tutor: \TUTOR}
\fancyhead[C]{\COURSE}
\fancyhead[R]{\today}

\fancyfoot[L]{}
\fancyfoot[C]{}
\fancyfoot[R]{Seite \thepage}

%Formatierung der Überschrift, hier nichts ändern
\def\header#1#2{
	\begin{center}
		{\Large\bf Übungsblatt #1}\\
		{(Abgabetermin #2)}
	\end{center}
}

%Definition der Punktetabelle, hier nichts ändern
\newcounter{punktelistectr}
\newcounter{punkte}
\newcommand{\punkteliste}[2]{%
	\setcounter{punkte}{#2}%
	\addtocounter{punkte}{-#1}%
	\stepcounter{punkte}%<-- also punkte = m-n+1 = Anzahl Spalten[1]
	\begin{center}%
		\begin{tabularx}{\linewidth}[]{@{}*{\thepunkte}{>{\centering\arraybackslash} X|}@{}>{\centering\arraybackslash}X}
			\forloop{punktelistectr}{#1}{\value{punktelistectr} < #2 } %
			{%
				\thepunktelistectr &
			}
			#2 &  $\Sigma$ \\
			\hline
			\forloop{punktelistectr}{#1}{\value{punktelistectr} < #2 } %
			{%
				&
			} &\\
			\forloop{punktelistectr}{#1}{\value{punktelistectr} < #2 } %
			{%
				&
			} &\\
		\end{tabularx}
	\end{center}
}

\begin{document}
	
	\begin{tabularx}{\linewidth}{m{0.2 \linewidth}X}
		\begin{minipage}{\linewidth}
			\STUDENTA\\
			\STUDENTB\\
			%\STUDENTC
		\end{minipage} & \begin{minipage}{\linewidth}
			\punkteliste{1}{\EXERCISES}
		\end{minipage}\\
	\end{tabularx}
	
	%\header{Nr. \NUMBER}{\DEADLINE}
	
	% ----------------------- TODO ---------------------------
	% Hier werden die Aufgaben/Lösungen eingetragen:
	
	

\section*{Augabe 1}
Sowohl die Entwicklung wie auch die Fertigung können Einfluss auf die Ausbeute nehmen. Die Entwicklung beispielsweiße über die IC größe (kleinere Chips führt zu mehr guten Chips) und die Fertigung zum Beispiel über reinere Rohmaterialien und präzisere Produktionsschritte.

\section*{Aufgabe 2}
Das Entwicklungsergebnis sind die Maskengeometrie, Tester für die Chips, Design und Spec-Parameter sowie die tatsächliche Funktionalität der IC's.


\section*{Aufgabe 3}
Am Ende der Waferfertigung werden die Spezifikationsparameter und die Funktion der IC's getestet. Zu testende Spezifikationsparameter sind:
\begin{itemize}
	\item Technologieparameter wie Schwellspannung der Transistoren, Kanallängen, Dotierung, usw.
	\item Spezifikationsparameter wie maximale Taktfrequenz, Leistungsaufnahme, Leckstrom, H- und L- Level mit jeweiligen Toleranzen, usw.
	\item Designparameter wie Transistor Größe, Schaltzeiten für Gatter, Widerstände für Leitungen, usw.
\end{itemize}


\section*{Aufgabe 4}
Bei der Herstellung sind Spot-Defects unvermeidlich, diese schädigen lokale Strukturen. Diese Punktdefekte führen zu einer nicht 100 Prozentigen Ausbeute.


\section*{Augabe 5}
Die Defektdichte gibt die Anzahl an Defekten pro Flächeneinheit 


\section*{Aufgabe 6}
Die Ausbeute von guten Chips ist der Prozentsatz der guten von den geometrisch auf dem Wafer entahltenen IC's.


\section*{Aufgabe 7}
Indem man die Defektdichte in der Fertigung reduziert. Dies ist möglich durch beispielsweiße reinere Materialien, genauerer Ätzung oder prinzipiell durch präzisere Fertigung  der Wafer mit reineren Rohmaterialien. Eine Weitere Möglichkeit ist die Fläche der IC's zu verringeren.


\section*{Aufgabe 8}
Durch die Variablen Chipfläche, der Defektdichte kann die Ausbeute berechnet werden.\\
Zur Berechnung der Defektdichte können entweder feste Werte oder Wahrscheinlichkeitsfunktionen, wobei dann die Fläche unter der jeweiligen Kurve )Moore, Murphy-Dirac, Murphy-Brickwall, Murphy-Dreieck, Seeds) die Defektdichte ist, genutzt werden.

\section*{Augabe 9}
$Y_w = e^{-\sqrt{AD}}$
Ist theoretisch in keiner Weise begründbar.


\section*{Aufgabe 10}

Murhpy-Dirac:\\
$
\frac{K_n}{K_a} = \frac{A_n}{A_a} \cdot e^{A_n \cdot D - A_a \cdot D}\\
~\\
D_0 = 1:\\
\frac{2}{1} \cdot e^{2 \cdot 1 - 1 \cdot 1} = 2e^1 = 5.436563\\
\\
D_0 = 0.5:\\
\frac{2}{1} \cdot e^{2 \cdot 0.5 - 0.5} = 2e^{0.5} = 1.21306
$\\\\
~\\
Murphy-Dreieck:\\
$
\frac{K_n}{K_a} = \frac{A_n}{A_a} \cdot \left(
	\frac{
		\left(\frac{1-e^{A_nD}}{A_nD}\right)^2
	}
{
	\left(\frac{1-e^{A_aD}}{A_aD}\right)^2
}
\right)\\
D_0 = 1:\\
= \frac{2}{1} \cdot \left(
\frac{
	\left(\frac{1-e^{2 \cdot 1}}{2 \cdot 1}\right)^2
}
{
	\left(\frac{1-e^{1 \cdot 1}}{1 \cdot 1}\right)^2
}
\right)\\
= \frac{2}{1} \cdot \left(
\frac{
	\left(\frac{1-e^{2}}{2}\right)^2
}
{
	\left(\frac{1-e^{1}}{1}\right)^2
}
\right)=6.91280
\\\\
~\\
D_0 = 0.5:\\
= \frac{2}{1} \cdot \left(
\frac{
	\left(\frac{1-e^{2 \cdot 0.5}}{2 \cdot 0.5}\right)^2
}
{
	\left(\frac{1-e^{1 \cdot 0.5}}{1 \cdot 0.5}\right)^2
}
\right)\\
= \frac{2}{1} \cdot \left(
\frac{
	\left(\frac{1-e^{1}}{1}\right)^2
}
{
	\left(\frac{1-e^{0.5}}{0.5}\right)^2
}
\right)= 3.50786
$
\end{document}
