 % ----------------------- TODO ---------------------------
%Template 
\documentclass[a4paper]{scrartcl}
\usepackage[utf8]{inputenc}
%\usepackage[ngerman]{babel}
\usepackage{geometry,forloop,fancyhdr,fancybox,lastpage}
\usepackage{listings}
\lstset{frame=tb,
	language=Java,
	aboveskip=3mm,
	belowskip=3mm,
	showstringspaces=false,
	columns=flexible,
	basicstyle={\small\ttfamily},
	numbers=left,
	numberstyle=\tiny\color{gray},
	keywordstyle=\color{blue},
	commentstyle=\color{dkgreen},
	stringstyle=\color{mauve},
	breaklines=true,
	breakatwhitespace=true,
	tabsize=3
}
\geometry{a4paper,left=3cm, right=3cm, top=3cm, bottom=3cm}
% Diese Daten müssen pro Blatt angepasst werden:
\newcommand{\NUMBER}{1}
\newcommand{\EXERCISES}{10}
% Diese Daten müssen einmalig pro Vorlesung angepasst werden:
\newcommand{\COURSE}{Chip Design}
%\newcommand{\TUTOR}{Benjamin Coban}
\newcommand{\STUDENTA}{Stefan Wezel}
\newcommand{\STUDENTB}{Lukas Günthner}
%\newcommand{\STUDENTC}{Gwent Krause}
\newcommand{\DEADLINE}{\date}
% ----------------------- TODO ---------------------------



%Math
\usepackage{amsmath,amssymb,tabularx}

%Figures
\usepackage{graphicx,tikz,color,float}
\graphicspath{ {home/stefan/picures/} }
\usetikzlibrary{shapes,trees}

%Algorithms
\usepackage[ruled,linesnumbered]{algorithm2e}

%Kopf- und Fußzeile
\pagestyle {fancy}
%\fancyhead[L]{Tutor: \TUTOR}
\fancyhead[C]{\COURSE}
\fancyhead[R]{\today}

\fancyfoot[L]{}
\fancyfoot[C]{}
\fancyfoot[R]{Seite \thepage}

%Formatierung der Überschrift, hier nichts ändern
\def\header#1#2{
	\begin{center}
		{\Large\bf Übungsblatt #1}\\
		{(Abgabetermin #2)}
	\end{center}
}

%Definition der Punktetabelle, hier nichts ändern
\newcounter{punktelistectr}
\newcounter{punkte}
\newcommand{\punkteliste}[2]{%
	\setcounter{punkte}{#2}%
	\addtocounter{punkte}{-#1}%
	\stepcounter{punkte}%<-- also punkte = m-n+1 = Anzahl Spalten[1]
	\begin{center}%
		\begin{tabularx}{\linewidth}[]{@{}*{\thepunkte}{>{\centering\arraybackslash} X|}@{}>{\centering\arraybackslash}X}
			\forloop{punktelistectr}{#1}{\value{punktelistectr} < #2 } %
			{%
				\thepunktelistectr &
			}
			#2 &  $\Sigma$ \\
			\hline
			\forloop{punktelistectr}{#1}{\value{punktelistectr} < #2 } %
			{%
				&
			} &\\
			\forloop{punktelistectr}{#1}{\value{punktelistectr} < #2 } %
			{%
				&
			} &\\
		\end{tabularx}
	\end{center}
}

\begin{document}
	
	\begin{tabularx}{\linewidth}{m{0.2 \linewidth}X}
		\begin{minipage}{\linewidth}
			\STUDENTA\\
			\STUDENTB\\
			%\STUDENTC
		\end{minipage} & \begin{minipage}{\linewidth}
			\punkteliste{1}{\EXERCISES}
		\end{minipage}\\
	\end{tabularx}
	
	%\header{Nr. \NUMBER}{\DEADLINE}
	
	% ----------------------- TODO ---------------------------
	% Hier werden die Aufgaben/Lösungen eingetragen:
	
	

\section*{Frage 1}
Das Zeitverhalten kann sich einmal durch die Beschreibung der Anstiegs- bzw Abfallszeit des Signals von 10\% bis 90\% des Signalhubs (o. Ä. je nach Spezifikation) beschrieben werden oder durch die Gatterverzögerungszeit, also Abstand der 50\% Punkte des Ausgangs- und Eingagnssignals.
~\\
~\\
\section*{Frage 2}
Die Gatter-Verzögerungszeit ist abhängig von:
\begin{itemize}
	\item Lastkapazität
	\item Treibfähigkeit des Bausteins
	\item Temperatur und Prozessparameter
	\item Versorgungsspannung am Gate
	\item Signalanstiegsgeschwindigkeit am Eingang
\end{itemize}
~\\
~\\
\section*{Frage 3}
Entweder durch anpassen des Verhältnis der Dimensionierung der P-MOS und N-MOS Transistoren oder durch anpassen der Versorgungsspannung.
~\\
~\\
\section*{Frage 4}
Als Miller-Kapazität bezeichnet man die Gate-Drain-Kapazitäten eines Verstärkers, also die Kapazitäten die vom Ausgang auf den Eingang des Verstärkers zurückführen. Diese erscheinen um den Betrag der Verstärkung größer.\\
Dieser Effekt ist meist unerwünscht, da er den Einsatz der Emitter- bzw. Sourceschaltungen auf niedrige bis mittlere Frequenzen beschränkt.
~\\
~\\
\section*{Frage 5}
Die SWITCH-LEVEL-Simulation wird zur vereinfachten Bestimmung der Schaltzeit verwendet. Dabei wird jeder Transistor als Schalter mit den jeweils entsprechenden Widerständenin Serie dargestellt. Der genau Betrag der Widerstände wird beispielsweiße über eine genaue Simulation festgelegt. Die Schalter schalten dann, wenn die steuernde Spannung die Schwellspannung des Invertres durchläuft.\\
Probleme beim SWITCH-LEVEL-Modell und -Simulation sind, dass der Querstrom durch das Gatter von der Versorgungsspannung zur Masse, sowie die Nichtlineariät der Transistoren. unbrücksichtig sind.
~\\
~\\
\section*{Frage 6}
Bei der TRANSISTOR-LEVEL-Simulation wird eine genaue Simulation durch Lösen der nichtlinearen gewöhnlichen Differntialgelichungen die die Transistoren beschreiben durchgeführt.
~\\
~\\
\section*{Frage 7}
Da bei unnötig groß dimensonierten Gattern der Energieaufwand steigt und damit auch die abwärme. Dementsprechend macht es Sinn Gatter immer entsprechend ihrer zu treibenden Kapazität zu dimensionieren, um so wenig Energie wie möglich zu verbrauchen (damit auch so wenig wie möglich abwärme zu produzieren) aber dennoch die Verzögerungszeit des Gatters der Anwendung entspricht um Fehlfunktionen zu vermeiden.
~\\
~\\
\section*{Frage 8}
Durch Inverterkaskaden können Treiber für hohe Lastkapazitäten entworfen werden. Beim Entwurf von Inverterkaskaden um hohe Lastkapazitäten zu betreiben ergeben sich zwei Frage. Zum einen wie viele Stufen sind einzubauen sodass man sich am Laufzeit-Optimum befindet und zum anderen wie die jeweiligen Transistoren in den Stufen zu dimensionieren sind.\\
Um die Laufzeit zu optimieren kann die Formel $t_{pges} = n \cdot A + B \cdot \sum_{i=1}^n \alpha_i$ mit $A = \frac{V_{dd} \cdot K_a}{K_t}$, $B = \frac{V_{dd} \cdot K_e}{Kt}$ und $\alpha_i = \frac{w_{i+1}}{w_i}$ verwendet werden.\\
Um dann die optimale Inverter-Anzahl $n$ zu bestimmen wird in $t_{pges} = n \cdot A + B \cdot (\frac{W_{n+1}}{W_0})^{\frac{1}{n+1}} $ das minimum gesucht.
Bei beiden Vorgängen wird die Lastkapazität durch die Weite $w_{n+1}$ dargestellt (also $\alpha = (\frac{W_{n+1}}{W_0})^{\frac{1}{n+1}}$).
~\\
~\\
\section*{Frage 9}
Da es alle Gatter für unterschiedliche Treiberfähigkeiten in verschiedenen Ausführungen gibt.
~\\
~\\
\section*{Frage 10}
Durch verringern der Versorgungsspannung erhöht sich die Verzögerungszeit (durch halbeieren der Versorgungsspannung ca eine Verdopplung der Verzögerungszeit siehe Folie 8-31).
~\\
~\\
\section*{Frage 11}
Mit steigender Temperatur sinkt die Beweglichkeit der Ladungsträger $\mu_n$ und $\mu_p$, somit steigt die Verzögerungszeit des Gatters.
~\\
~\\
\end{document}
