 % ----------------------- TODO ---------------------------
%Template 
\documentclass[a4paper]{scrartcl}
\usepackage[utf8]{inputenc}
%\usepackage[ngerman]{babel}
\usepackage{geometry,forloop,fancyhdr,fancybox,lastpage}
\usepackage{listings}
\lstset{frame=tb,
	language=Java,
	aboveskip=3mm,
	belowskip=3mm,
	showstringspaces=false,
	columns=flexible,
	basicstyle={\small\ttfamily},
	numbers=left,
	numberstyle=\tiny\color{gray},
	keywordstyle=\color{blue},
	commentstyle=\color{dkgreen},
	stringstyle=\color{mauve},
	breaklines=true,
	breakatwhitespace=true,
	tabsize=3
}
\geometry{a4paper,left=3cm, right=3cm, top=3cm, bottom=3cm}
% Diese Daten müssen pro Blatt angepasst werden:
\newcommand{\NUMBER}{1}
\newcommand{\EXERCISES}{16}
% Diese Daten müssen einmalig pro Vorlesung angepasst werden:
\newcommand{\COURSE}{Chip Design}
%\newcommand{\TUTOR}{Benjamin Coban}
\newcommand{\STUDENTA}{Stefan Wezel}
\newcommand{\STUDENTB}{Lukas Günthner}
%\newcommand{\STUDENTC}{Gwent Krause}
\newcommand{\DEADLINE}{\date}
% ----------------------- TODO ---------------------------



%Math
\usepackage{amsmath,amssymb,tabularx}

%Figures
\usepackage{graphicx,tikz,color,float}
\graphicspath{ {home/stefan/picures/} }
\usetikzlibrary{shapes,trees}

%Algorithms
\usepackage[ruled,linesnumbered]{algorithm2e}

%Kopf- und Fußzeile
\pagestyle {fancy}
%\fancyhead[L]{Tutor: \TUTOR}
\fancyhead[C]{\COURSE}
\fancyhead[R]{\today}

\fancyfoot[L]{}
\fancyfoot[C]{}
\fancyfoot[R]{Seite \thepage}

%Formatierung der Überschrift, hier nichts ändern
\def\header#1#2{
	\begin{center}
		{\Large\bf Übungsblatt #1}\\
		{(Abgabetermin #2)}
	\end{center}
}

%Definition der Punktetabelle, hier nichts ändern
\newcounter{punktelistectr}
\newcounter{punkte}
\newcommand{\punkteliste}[2]{%
	\setcounter{punkte}{#2}%
	\addtocounter{punkte}{-#1}%
	\stepcounter{punkte}%<-- also punkte = m-n+1 = Anzahl Spalten[1]
	\begin{center}%
		\begin{tabularx}{\linewidth}[]{@{}*{\thepunkte}{>{\centering\arraybackslash} X|}@{}>{\centering\arraybackslash}X}
			\forloop{punktelistectr}{#1}{\value{punktelistectr} < #2 } %
			{%
				\thepunktelistectr &
			}
			#2 &  $\Sigma$ \\
			\hline
			\forloop{punktelistectr}{#1}{\value{punktelistectr} < #2 } %
			{%
				&
			} &\\
			\forloop{punktelistectr}{#1}{\value{punktelistectr} < #2 } %
			{%
				&
			} &\\
		\end{tabularx}
	\end{center}
}

\begin{document}
	
	\begin{tabularx}{\linewidth}{m{0.2 \linewidth}X}
		\begin{minipage}{\linewidth}
			\STUDENTA\\
			\STUDENTB\\
			%\STUDENTC
		\end{minipage} & \begin{minipage}{\linewidth}
			\punkteliste{1}{\EXERCISES}
		\end{minipage}\\
	\end{tabularx}
	
	%\header{Nr. \NUMBER}{\DEADLINE}
	
	% ----------------------- TODO ---------------------------
	% Hier werden die Aufgaben/Lösungen eingetragen:
	
	

\section*{Frage 1}

~\\
~\\
\section*{Frage 2}

~\\
~\\
\section*{Frage 3}
Da mit MS-FF sichergestellt werden kann, dass die Ausgäge zu einem festgelegten Zeitpunkt für alle FF gleichzteig geschalten werden. Bei einfachen latches können die Ausgänge transparent werden, durch die Rückkopplung würde hier eventuell die vorhergehende Berechnung beeinflusst.
~\\
~\\
\section*{Frage 4}
FF mit asynchronem Reset werden Beispielweiße bei Power-On-Reset Schaltungen verwendet.
~\\
~\\
\section*{Frage 5}
Die Power-On-Schatung initalisiert das Schaltwerk in einen definierten Startzustand. Sobald beim ansteigen der $V_{dd}$ die minimale $V_{dd}$ für den POR erreicht ist wird er taktasynchron aktiv, zu dieser Zeit ist das Taktsignal noch instabil und der POR zieht das Taktzignal bspw. auf $0V$. Sobald das Taktsignal stabil und die $V_{dd}$ stabil sind, wird der PORT taktsynchron inaktiv und die Schaltung beginnt mit dem stabilen Takt.
~\\
~\\
\section*{Frage 6}
Im asynchronen Schaltwerk wird der Zustand praktisch "auf der Leitung" mit zusätzlichen Rückkopplungen gespeichert, hier ist es extrem wichtig auf die Laufzeitbedinungen zu achten! Es gibt keinen Zustands-Speicher mit Latches.
~\\
~\\
\section*{Frage 7}
Da durch das Zeitverhalten der Takt über das Taktverteilungssystem zu verschiedenen tatsächlichen Zeitpunkten an verschiedenen Punkten in der Schaltung ankommen kann. Dies kann zu berechnungsfehler führen wenn bspw. zwei Gatter in verschiedenen Taktzyklen arbeiten.
~\\
~\\
\section*{Frage 8}
\begin{itemize}
	\item Die abtastende Taktflanke muss an allen FFs gleichzeitig eintreffen.
	\item Die zeitlichen Interveralle zwischen Taktflanken müssen ausreichen sodass das Schaltnetz genug Zeit hat sich auf die tatsächliche Ausgangssignale einschingen kann.
	\item Die Signale am Eingang der FFs müssen um den Abtastzeitpunkt stabile und gültige Logikpegel sein.
\end{itemize}
~\\
~\\
\section*{Frage 9}

~\\
~\\
\section*{Frage 10}
STA ist die statische Laufzeitanalyse. Es werden alle Pfade im Schaltnetz berücksichtigt, auch logisch nicht schaltbare Pfade. Es wird unter Verwendung der beschriebenen Laufzeitmodelle für die entsprechenden Zellen, die Laufzeit für jeden Pfad für sowhol steigende als auch für die fallende Taktflanke berechnet. Hierfür wird auch die zu treibende Kapazität für jedes Gatter benötigt, diese ergibt sich aus der Summe der Eingangskapazitäten aller folgenden Gatter auf dem Pfad plus die Kapazität der Verbindungsleitungen.\\
Als Pfadlauzeit wird dann aus der Laufzeit für die steigende und fallende Taktflanke die größere ausgewählt.\\
Da die STA vor dem Layout durchgeführt wird, wählt man ein Schätzverfahren für die Verdrahtungskapazitäten. Nach der Layout erzeugung muss die STA wiederholt werden.\\
Es kann zu mehreren Iterationen der STA kommen, da bei einer Änderung auf einem Pfad, die Laufzeit für alle beeinflussten Pfaden neu berechnet werden muss und sich hierdurch wieder andere Pfade zu kritischen Pfaden werden können.\\
Als weiteres Modell der STA gibt es die Statistical Static Timing Analysis, bei der die STA mit statistischen Methoden erweitert wird. So soll die Intra-Chip-Streuung der Laufzeiten der einzelnen Gatter besser erfasst werden.\\
~\\
~\\
\section*{Frage 11}
Ein kritischer Pfad ist ein Pfad im Schaltnetz, der kaum noch oder keine Laufzeit-Reserver mehr hat.
~\\
~\\
\section*{Frage 12}

~\\
~\\
\section*{Frage 13}
Es ist zu beachten, dass der Bus an den die Tri-State Treiber angeschlossen sind immer ein definierter Logikzustand anliegt, das heißt dass entweder durch ein TG zur Bus-Definition der Bus einen definierten und gültigen Logikzustand hat, oder dass immer genau ein Tri-State Treiber ein gültiges Signal auf den Bus schält.
~\\
~\\
\section*{Frage 14}

~\\
~\\
\section*{Frage 15}
Da ein TG nicht verstärkt und nicht entkoppelt, können Latches vor dem TG durch die Lastkapazität "rückwärts" beeinflusst werden.\\
Dies kann bspw. bei Bus-Treiber problematisch werden. Wenn das Ausgangssignal eines Latches high ist und durch ein TG auf den Bus geschaltet wird, erhöht sich die Ausgangskapazität um die Kapazität des Busses. Nun kann es passieren, dass die High-Ausgangsspannung nicht genügt um die neue große Kapazität auf High umzuladen und so der Ausgang als Low-Pegel interpretiert wird und das Latch über die Rückkopplung umgeworfen wird.
~\\
~\\
\section*{Frage 16}

~\\
~\\

\end{document}
