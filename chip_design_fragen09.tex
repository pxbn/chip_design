 % ----------------------- TODO ---------------------------
%Template 
\documentclass[a4paper]{scrartcl}
\usepackage[utf8]{inputenc}
%\usepackage[ngerman]{babel}
\usepackage{geometry,forloop,fancyhdr,fancybox,lastpage}
\usepackage{listings}
\lstset{frame=tb,
	language=Java,
	aboveskip=3mm,
	belowskip=3mm,
	showstringspaces=false,
	columns=flexible,
	basicstyle={\small\ttfamily},
	numbers=left,
	numberstyle=\tiny\color{gray},
	keywordstyle=\color{blue},
	commentstyle=\color{dkgreen},
	stringstyle=\color{mauve},
	breaklines=true,
	breakatwhitespace=true,
	tabsize=3
}
\geometry{a4paper,left=3cm, right=3cm, top=3cm, bottom=3cm}
% Diese Daten müssen pro Blatt angepasst werden:
\newcommand{\NUMBER}{1}
\newcommand{\EXERCISES}{7}
% Diese Daten müssen einmalig pro Vorlesung angepasst werden:
\newcommand{\COURSE}{Chip Design}
%\newcommand{\TUTOR}{Benjamin Coban}
\newcommand{\STUDENTA}{Stefan Wezel}
\newcommand{\STUDENTB}{Lukas Günthner}
%\newcommand{\STUDENTC}{Gwent Krause}
\newcommand{\DEADLINE}{\date}
% ----------------------- TODO ---------------------------



%Math
\usepackage{amsmath,amssymb,tabularx}

%Figures
\usepackage{graphicx,tikz,color,float}
\graphicspath{ {home/stefan/picures/} }
\usetikzlibrary{shapes,trees}

%Algorithms
\usepackage[ruled,linesnumbered]{algorithm2e}

%Kopf- und Fußzeile
\pagestyle {fancy}
%\fancyhead[L]{Tutor: \TUTOR}
\fancyhead[C]{\COURSE}
\fancyhead[R]{\today}

\fancyfoot[L]{}
\fancyfoot[C]{}
\fancyfoot[R]{Seite \thepage}

%Formatierung der Überschrift, hier nichts ändern
\def\header#1#2{
	\begin{center}
		{\Large\bf Übungsblatt #1}\\
		{(Abgabetermin #2)}
	\end{center}
}

%Definition der Punktetabelle, hier nichts ändern
\newcounter{punktelistectr}
\newcounter{punkte}
\newcommand{\punkteliste}[2]{%
	\setcounter{punkte}{#2}%
	\addtocounter{punkte}{-#1}%
	\stepcounter{punkte}%<-- also punkte = m-n+1 = Anzahl Spalten[1]
	\begin{center}%
		\begin{tabularx}{\linewidth}[]{@{}*{\thepunkte}{>{\centering\arraybackslash} X|}@{}>{\centering\arraybackslash}X}
			\forloop{punktelistectr}{#1}{\value{punktelistectr} < #2 } %
			{%
				\thepunktelistectr &
			}
			#2 &  $\Sigma$ \\
			\hline
			\forloop{punktelistectr}{#1}{\value{punktelistectr} < #2 } %
			{%
				&
			} &\\
			\forloop{punktelistectr}{#1}{\value{punktelistectr} < #2 } %
			{%
				&
			} &\\
		\end{tabularx}
	\end{center}
}

\begin{document}
	
	\begin{tabularx}{\linewidth}{m{0.2 \linewidth}X}
		\begin{minipage}{\linewidth}
			\STUDENTA\\
			\STUDENTB\\
			%\STUDENTC
		\end{minipage} & \begin{minipage}{\linewidth}
			\punkteliste{1}{\EXERCISES}
		\end{minipage}\\
	\end{tabularx}
	
	%\header{Nr. \NUMBER}{\DEADLINE}
	
	% ----------------------- TODO ---------------------------
	% Hier werden die Aufgaben/Lösungen eingetragen:
	
	

\section*{Frage 1}
Hauptanwendung von Transfer-Gates (TG) sind Mulitplexer, da Multiplexer aus TG schneller sind als äquivalente Multiplexer aus CMOS NAND/NOR Gatter.
~\\
~\\
\section*{Frage 2}
Zum einen verstärkt ein TG nicht, desweiteren muss bei TG darauf geachtet werden, dass der Ausgang niemals undefiniert ist wenn ein TG nicht leitend ist (offener Schalter). Andere Logigatter hingegen können auch als Verstärker verwendet werden und haben auf Grund ihrer Bauweise niemals ein undefinierten Zustand.
~\\
~\\
\section*{Frage 3}
Transfer Gates "regenerieren" den Logikpegel nicht, daher sind Treiber notwendig. Diese können bspw. ein Inverter oder andere Logikgatter vor dem Ausgang einer Zelle sein.\\
Außerdem dürfen die Inputs dieser Treiber niemals undefiniert sein, daher muss im TG immer ein leitender Weg vom Ausgang einer Verknüpfungsschaltung zu einem normal getriebenen Eingang existieren.
~\\
~\\
\section*{Frage 4}
Vorteile sind beispielsweiße, dass weniger Transistoren als bei standart CMOS Schaltungen verwendet werden können und somit auch die Anzahl an parasitären Tansistoren kleiner ist. Desweiteren führt die Kombination von PMOS und NMOS im TG dazu, dass die noise margin nich verkleinert wird und es keinen steigenden Umschalt Widerstand gibt.
Nachteile sind zum einen, dass TG langsamer als CMOS Gatter sind und daraus folgend können aufgrund des Laufzeitverhaltens Kurzschlüsse entsehen. Zuletzt können aufgrund von der Verteilung der Ladung nicht mehr als ca 3 TG in Reihe geschalten werden.
~\\
~\\
\section*{Frage 5}
Der On-Leitwert sinkt mit Zunehmener Spannung, damit folgt auch dass der On-Widerstand mit zunehmender Spannung Steigt. Der Nachteil ist offensichtlich, dass ein TG nur mit N-MOS Transistor nicht in höheren Spannungsbereichen betrieben werden kann. Auf der anderen seite ist der Vorteil, dass wenn die Schaltung sich sowieso nur in niedrigen Spannungsbereichen befindet, wir ein Transistor sparen können.
~\\
~\\
\section*{Frage 6}
Die Gestaltung  der Modellierung der Gatterlaufzeit ist bei TGs wesentlich komplizierter. Herkömmliche Platzierungs- und Verbindungsverfahren können hierbei nicht mehr genutzt werden wenn das Treiben der Last durch ein TG an eine andere Zelle "weitergereicht" wird.\\
\\
Deshalb muss die Lastkapazität durch ein Gatter am Ausgang der TG-Standartzelle abgetrennt werden. So erhält man zum einen die parasitäre Kapazität der TG-Zelle und zum anderen hinter dem bspw. Inverter die Lastkapazität.
~\\
~\\
\section*{Frage 7}

~\\
~\\
\end{document}
