 % ----------------------- TODO ---------------------------
%Template 
\documentclass[a4paper]{scrartcl}
\usepackage[utf8]{inputenc}
%\usepackage[ngerman]{babel}
\usepackage{geometry,forloop,fancyhdr,fancybox,lastpage}
\usepackage{listings}
\lstset{frame=tb,
	language=Java,
	aboveskip=3mm,
	belowskip=3mm,
	showstringspaces=false,
	columns=flexible,
	basicstyle={\small\ttfamily},
	numbers=left,
	numberstyle=\tiny\color{gray},
	keywordstyle=\color{blue},
	commentstyle=\color{dkgreen},
	stringstyle=\color{mauve},
	breaklines=true,
	breakatwhitespace=true,
	tabsize=3
}
\geometry{a4paper,left=3cm, right=3cm, top=3cm, bottom=3cm}
% Diese Daten müssen pro Blatt angepasst werden:
\newcommand{\NUMBER}{1}
\newcommand{\EXERCISES}{11}
% Diese Daten müssen einmalig pro Vorlesung angepasst werden:
\newcommand{\COURSE}{Chip Design}
%\newcommand{\TUTOR}{Benjamin Coban}
\newcommand{\STUDENTA}{Stefan Wezel}
\newcommand{\STUDENTB}{Lukas Günthner}
%\newcommand{\STUDENTC}{Gwent Krause}
\newcommand{\DEADLINE}{\date}
% ----------------------- TODO ---------------------------



%Math
\usepackage{amsmath,amssymb,tabularx}

%Figures
\usepackage{graphicx,tikz,color,float}
\graphicspath{ {home/stefan/picures/} }
\usetikzlibrary{shapes,trees}

%Algorithms
\usepackage[ruled,linesnumbered]{algorithm2e}

%Kopf- und Fußzeile
\pagestyle {fancy}
%\fancyhead[L]{Tutor: \TUTOR}
\fancyhead[C]{\COURSE}
\fancyhead[R]{\today}

\fancyfoot[L]{}
\fancyfoot[C]{}
\fancyfoot[R]{Seite \thepage}

%Formatierung der Überschrift, hier nichts ändern
\def\header#1#2{
	\begin{center}
		{\Large\bf Übungsblatt #1}\\
		{(Abgabetermin #2)}
	\end{center}
}

%Definition der Punktetabelle, hier nichts ändern
\newcounter{punktelistectr}
\newcounter{punkte}
\newcommand{\punkteliste}[2]{%
	\setcounter{punkte}{#2}%
	\addtocounter{punkte}{-#1}%
	\stepcounter{punkte}%<-- also punkte = m-n+1 = Anzahl Spalten[1]
	\begin{center}%
		\begin{tabularx}{\linewidth}[]{@{}*{\thepunkte}{>{\centering\arraybackslash} X|}@{}>{\centering\arraybackslash}X}
			\forloop{punktelistectr}{#1}{\value{punktelistectr} < #2 } %
			{%
				\thepunktelistectr &
			}
			#2 &  $\Sigma$ \\
			\hline
			\forloop{punktelistectr}{#1}{\value{punktelistectr} < #2 } %
			{%
				&
			} &\\
			\forloop{punktelistectr}{#1}{\value{punktelistectr} < #2 } %
			{%
				&
			} &\\
		\end{tabularx}
	\end{center}
}

\begin{document}
	
	\begin{tabularx}{\linewidth}{m{0.2 \linewidth}X}
		\begin{minipage}{\linewidth}
			\STUDENTA\\
			\STUDENTB\\
			%\STUDENTC
		\end{minipage} & \begin{minipage}{\linewidth}
			\punkteliste{1}{\EXERCISES}
		\end{minipage}\\
	\end{tabularx}
	
	%\header{Nr. \NUMBER}{\DEADLINE}
	
	% ----------------------- TODO ---------------------------
	% Hier werden die Aufgaben/Lösungen eingetragen:
	
	

\section*{Frage 1}
Die Inversion ist der Zustand einer MOS-Halbleiter-Struktur, bei dem die Angelegte Spannung größer als die Schwellspannung $V_t < V_g$ ist. An der Oberfläche des Siliciums versammelen sich eine Schicht aus Minoritätsträgern, die dort durch ihre hohe Konzentration den Leitfähigkeitstyp umkehren.
~\\
~\\
\section*{Frage 2}
%TODO:Verstehe frage nicht
~\\
~\\
\section*{Frage 3}
Die Akkumulation entsteht wenn die angelegte Spannung kleiner als die Flachbandspannung $V_g < V_{FB}$ ist. Hier bewegen sich die Majoritätsladungsträger an die Oberfläche des Siliciums und es entseht eine Schicht die sich wie eine metallisch leitende Platte verhält.
~\\
~\\
\section*{Frage 4}
Die Verarmung einer MOS-Halbleiter-Struktur entsteht wenn die angelegte Spannung zwischen der Flachbandspannung und der Schwellspannung liegt, also $V_{FB} \le V \le V_t$. Es kommt zu einer Verringerung der Majoritätsladungsträger an der Oberfläche des Siliciums. Wichtig ist hierbei jedoch, dass es keine beweglichen Ladungsträger gibt die zur Stromleitung zur Verfügung stehen. 
~\\
~\\
\section*{Frage 5}
Die Flachbandspannung ist die Spannung, bei der die gespeicherte Ladung im Halbleiter null wird und die Energiebänder dementsprechend Flach sind. Die Verhältnisse an der Oberfläche sind also genauso wie im Inneren des Halbleiters.
~\\
~\\
\section*{Frage 6}
Nein
~\\
~\\
\section*{Frage 7}

~\\
~\\
\section*{Frage 8}
Im Halbleiter gibt es nur recht wenige bewegliche Ladungsträger, im Gegensatz zum Metall in dem es sehr viele Ladungsträger gibt und zum Isolator in dem es fast keine beweglichen Ladungsträger gibt. Desweitern ist es beim Halbleiter technisch "einfach" möglich die Konzentration an Ladungsträgern mithilfe von Dotierung zu verändern.
~\\
~\\
\section*{Frage 9}
In Metall stehen ca. $10^{22} \text{ Elektronen pro cm}^3$ zur Verfügung.
~\\
~\\
\section*{Frage 10}
Diamant-Gitter. Die nächsten Nachbar-Atome bilden eine Tetraeder-Strukur.
~\\
~\\
\section*{Frage 11}
N-Mos Transistoren können als Schalter gegen Masse verwendet werden und p-Mos Transistoren als Schalter gegen die positive Versorgungsspannung.
~\\
~\\
\section*{Frage 12}

~\\
~\\
\section*{Frage 13}

~\\
~\\
\section*{Frage 14}

~\\
~\\
\section*{Frage 15}

~\\
~\\
\section*{Frage 16}

~\\
~\\
\section*{Frage 17}

~\\
~\\
\section*{Frage 18}

~\\
~\\

\end{document}
